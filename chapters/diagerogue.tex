\chapter{Rouge}
\label{ch:rouge}
\begin{quotation}
\textsl{You are a nameless traveller that has been compelled to search for something.
An object, a person, or just a place.
You don't know yet, but it's up to you to find out.
You will travel through various caves, forests and towns to pursue the whims of your heart.
Every time you seem to solve a problem, another one turns up.
Always moving you towards some inevitable doom.
Are you born for heroism, or will you die unloved and unknown? \\\\Choose your path, and see where the voices in your head take you.}
\end{quotation}
Rouge is a \rogue tile-based game that is created specifically for the demonstration of the \diage narrative generation system.
Created within my own framework \textit{SilicaLib} created on top of the XNA game-framework created by Microsoft.
Rouge is characterised by the fact that the world and the narrative is generated by the direct influence of the player.

\section{Mechanics}
The game mechanics in Rouge are fully implemented within the \diage attribute system.
When ever actors interact with the world their attributes can effect the world according to their rules.
Whenever they interact with other actors, these attributes will influence dialogue or cause either an antagonistic, friendly or neutral response in accordance to their moral alignment.

Actors move around the world in turns.
When an entity gets the turn, its actions points are restored and he can use these to either move around the game world or attack an enemy.
Attacking an other actor results in a damage step.
This step subtracts the strength attribute of the attacker from the defence attribute and subtracts that outcome from the defenders health attribute.


Normal interaction with other actors does not cost an action point and resolves in a simple conversation.
The player asks the other actor for any information regarding his current quest.
If the actor knows about it, it will give either helpful or hurtful information, depending upon his alignment attribute.
The alignment resolution can also result in the actor refusing to answering the player's question if their alignments are to far apart from each other.

