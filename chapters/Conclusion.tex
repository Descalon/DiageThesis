\chapter*{Conclusion}
When I started my research I asked the questions \textit{How does the development process of a \rogue game benefit from using a generative narrative?}.

Procedural content is a large part of the \rogue experience. 
Without novel levels and a new way to traverse the world on every play through, the concept of turn-based dungeon crawling with perma-death loses its appeal.

In general terms; I hold that procedural narrative has value within video game development as a whole.
In 20 weeks I've built a system that enables a developer to create games that generate an emergent narrative as a result of simple values given to simple representations.
In chapter~\ref{ch:planning} I reasoned that for my scope and time going forward with a narrative planner was the way to go, in contrast to a BDI system.
A narrative planner gives me much more control over the narrative system, whereas a BDI model would not. 
My conclusion is, however, that any further development of the narrative system should contain a form of the BDI model, with the Narrator as overall director of the actors. 
While true that my research focuses on one type of game, the field is open for further work into other genres.
The Narrator itself can help developers create \rogue games within a smaller time-frame, due to the fact that the Narrator handles most of the story-writing.
Using the Narrator a designer could design a game around the ambiguity of the context, with the inherent fact that every play through can give a player a sense of an actual new adventure.
Even without the use of the Narrator, the attribute system allows designers to quickly design and test any game that has a implicit use of numbers, such as RPGs, Dungeon Crawlers and Tactics games.

As a summary; the development process gets the added benefit of a quick way to prototype stories in the form of the DML diagrams, a fast and efficient way of adding and perfecting game mechanics by the use of the narrative attribute system, and gains the ability to generate unique and novel narratives that is highly controllable, but requires relatively little input.