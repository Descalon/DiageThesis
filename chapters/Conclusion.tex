\chapter{Conclusion}
In this chapter I conclude my research and give some ideas for future study.\\

When I started my research, I asked if \rogue games can benefit from a procedurally generated narrative. I ask these questions from a designers point of view; \textit{"Does the game get better/more interesting?"}, and from a developers perspective: \textit{"Can we develop games faster/more efficiently"}. In the main text I gave multiple answers to these questions and concluded that they both have merit.

I hold that procedural narrative has value within video games. In 20 weeks I've built a system that enables a developer to create games that generate an emergent narrative as a result of simple values given to simple representations. While true that my research focuses on one type of game, the field is open for further work into other genres. \diage itself can help developers build \rogue games within a smaller time-frame, due to the fact that \diage handles most of the story-writing. 

Using \diage a designer could design a game around the ambiguity of the context, with the inherent fact that every play through can give a player a sense of an actual new adventure. Even without the use of \diage as a narrator, the attribute system allows designers to quickly design and test any game that has a implicit use of numbers, such as RPGs, Dungeon Crawlers and Tactics games. 

\section{Further Study}
As mentioned before, \diage was limited to \rogue games for the sake of scope. Other video game genres might be able to benefit from the use of a narrative system; especially games that are usually narrative-heavy.

\diage is normally expressed with a DML diagram, but the underlying structure is that of a graph. Little work has gone in to using this graph as a means to structure the direction of a narrative. More research in respect to graph translations in relation to \diage could result in a better way to direct the narrative into a specific direction. As example; the designer could define an initial state, and an eventual state. Using graph translations \diage could interpolate several \textit{in-between} states that the narrative will take. A method could be developed in which \diage could output all variants of narrative trees respectively to the choices available to the player. Displaying all possible endings a discreet story arc could have. 

It has been pointed out to me that the \diage system could be implemented in such a way that it could be used by the \textit{Mechanations} system developed by Dr. Joris Dormans\footnote{A bit too butt-kissy?} \ldots