\chapter{Reflection}
This section will cover my reflection upon the process of creating this document and all the work that relates to it in the style of a post-mortem.

\section{Diage}
Diage stands for \textit{Dialogue Generator}, which was the first idea that I had when starting this project. I wanted to make a generator that would look at the world state and create a dialogue that had impact on the context that the player was in at this moment. After several weeks of reading up on the related work, I switched the project to a more full narrative generative system. This was in part of it seeming more interesting and also somewhat easier to accomplish within the time frame I had. I really like the name \diage, so I stuck to it, even if it did lose it's meaning.

\section{The literature}
As previously mentioned; I spent weeks on reading literature. I've spent the better part of 10~12 weeks reading up on related work and developing the \textit{Diage Modelling Language} as a frame of reference. While the amount of reading is not unprecedented in a research study, it did bring the project in danger of finishing on time. After I did start working on what would become the Narrator I had misjudged some aspects of the event system which resulted in having to throw away a lot of work in early December, with less than 8 weeks until the deadline.