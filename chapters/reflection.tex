\chapter*{Reflection}
I feel that a good grasp on agile software development strengthens any developer in his work.
In any team effort I will always try to ensure that all team members have the same idea on how the development process should go, but not so much in any solo project.
I have the impression that my efforts on solo projects are less disciplined than when in a team and as such decided to use a more agile approach in the form of a \textit{design - develop - test - repeat} iterative process to try and structure my own projects.
\subsection*{What went wrong?}
When we started this project, I had no idea of the field I was getting into. 
I had some previous knowledge on the world of Interactive Storytelling, but not to the extent of my own research goals.
As a result, I spend a lot of time reading paper after paper on the idea of generating an interactive story.
With half of the project time spend reading, there wasn't a lot of time left to build and test my theories on how to implement a storytelling algorithm for video games, resulting in me pressing for time to get the product and my thesis done.
To make matters worse, during the last month of the project I decided to take on a completely different tract with my research, throwing away most of the work done already so I could do it properly.
This enormous set-back was the result of figuring out what kind of product I wanted to make, with the consequence that I was just trying a lot of things out and became jumbled.
\subsection*{What went right?}
I've always believed that good software comes from good processes, and this project was a great way to ingrain that in to my way of working. 
Instead of thinking of the product as a whole, I started to focus on what the functionalities where that I needed.
Each functionality was first designed and developed to work alone, and only when this tested positive it was included within the greater whole.
with no concrete idea of what the end result would be, I had no distractions as to work towards an actual product.
We had a feature-set that stated what we wanted to accomplish, and that was my guidance for the project.
\subsection*{Conclusion}
This method of working really suited me and helped me focus on the task at hand, without losing sight of the bigger picture.
I am of the opinion that software development could benefit a lot with this form of looking at a product.
Developers of a product can have a general idea for what the end result could be, but the spotlight should be on the requirements at hand, and solving them one by one until a product comes out of it. 